\documentclass[preprint,aps,prd,nofootinbib,superscriptaddress]{revtex4-2}

\usepackage[T1]{fontenc}
\usepackage[latin9]{inputenc}
\usepackage{amsmath}
\usepackage{graphicx}
%\usepackage{tikz-feynman}
\usepackage{simplewick}
\usepackage{slashed}
\usepackage{caption}
\usepackage{multirow}
\usepackage{float}
\usepackage{hyperref}
\usepackage[caption = false]{subfig}
\usepackage{mathtools}
\usepackage{tikz}
\usetikzlibrary{quantikz}

%%%%%%%%%%%%%%%%%%%%%%%%%%%%%% User specified LaTeX commands.

\newcommand{\BraKet}[2]{\langle #1 | #2 \rangle}

\begin{document}

%%%%%%%%%%%%%%%%%%%%%%%%%%%%%%%%%%%
\title{\boldmath Bernstein Vazirani Algorithm}


\author{Cody M. Grant}


\date{\today}

%
\maketitle
\newpage

%%%%%%%%%%%%%%%%%%%%%%%%%%%%%%%%%%%
\section{Problem}
%
This is an extension to the Deutsch-Jozsa alg (DJ alg). Assuming a black box function, $f$, which takes an input of a string of bits, it returns either 0 or 1:
%
\begin{eqnarray}
f(x_0, x_1, x_2) = 0 \text{ or } 1 \text{, where $x_n$ is 0 or 1}
\end{eqnarray}
%
However, instead of the function as balanced or constant, the function now guarantees to return a bitwise product of the input with some string, $s$:
%
\begin{eqnarray}
f(x) = s \cdot x (\text{mod 2})
\end{eqnarray}
%
We are expected to find $s$.


%%%%%%%%%%%%%%%%%%%%%%%%%%%%%%%%%%%
\section{Generic Quantum Solution}
%
Similarly to the DJ alg, we have the circuit:

\begin{figure} [H]
\centering
\begin{quantikz}
\lstick{$\ket{0^{\otimes n}}$} 	& \qwbundle{n}\qw	& \gate{H^{\otimes n}}\slice{1}
& \gate{f} \vqw{1} \slice{2} 	& \gate{H^{\otimes n}}\slice{3} 	& \qw	& \rstick{$\ket{S}$} \\
\lstick{$\ket{1}$} 				& \qw				& \gate{H}		
& \targ{}				& \qw 								& \qw	
\end{quantikz}
\end{figure}

First, we apply the Hadamard gate to all initial qubits
%
\begin{eqnarray}
\ket{\psi_1} &=& (H \otimes H)(\ket{1} \otimes \ket{0^{\otimes n}}) = \left[\frac{\ket{0} - \ket{1}}{\sqrt{2}}\right] \frac{1}{\sqrt{2^n}} \sum \limits_{x=0}^{2^n - 1}\ket{x}
\end{eqnarray}
%
For the classical case, the oracle $f_s$ returns 1 for any input $x$ such that $s \cdot x \text{ mod } 2=1$, and zero otherwise. Using the same phase kickback trick we did in the DJ alg and acting on the last qubit in the $\ket{-}$, we would have
%
\begin{eqnarray}
\ket{x} &\xrightarrow[f_s]{} & (-1)^{s \cdot x} \ket{x}
\end{eqnarray}
%
Similarly, quantum oracle would be
%
\begin{eqnarray}
\ket{\psi_2} &=& \frac{\ket{-}}{\sqrt{2^n}} \sum \limits_{x=0}^{2^n - 1}(-1)^{s \cdot x} \ket{x}
\end{eqnarray}
%
Like in the DJ alg, the last qubit in the $\ket{-}$ is only there to give us this extra factor of $(-1)^{s \cdot x}$. After applying this factor, a Hadamard gate to all but the last qubit, we can find the string, $s$, that was applied as it will be our final string (ignoring the last qubit). First I would like to note that 
%
\begin{eqnarray}
H^{\otimes n} {\sqrt{2^n}} \sum \limits_{x=0}^{2^n - 1}(-1)^{s \cdot x} \ket{x} &=& \frac{\ket{-}}{\sqrt{2^n}} \sum \limits_{x=0}^{2^n - 1}(-1)^{s \cdot x} \ket{x} = \ket{s}
\end{eqnarray}
%
Therefore, applying the last $H$ gate means:
%
\begin{eqnarray}
\ket{\psi_3} &=& \left(I \ket{-}\right)\otimes \left(H \left[\frac{1}{\sqrt{2^n}} \sum \limits_{x=0}^{2^n - 1}(-1)^{s \cdot x} \ket{x} \right]\right) = \ket{-}\otimes\ket{S}
\end{eqnarray}
%



%%%%%%%%%%%%%%%%%%%%%%%%%%%%%%%%%%%
\begin{thebibliography}{99}



\end{thebibliography}

\end{document}


